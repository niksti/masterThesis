\documentclass{cacthesis}

%Define your additional packages here
\usepackage{xcolor}

\begin{document}

	\frontmatter
	
	%%%%%%%%%%%%%
	%% Title page
	%%%%%%%%%%%%%
	\title{Incoercible Sale}
	\subtitle{Longer subtitle (if required)}	%Delete subtitle if not required
	\author{Nikolaos Stivaktakis}
	\date{September 26, 2019}
	\subject{Master Thesis}
	
	\publishers{
		\small
		\begin{tabular}{r l}
			Supervisors:  & Prof. Sebastian Faust, Ph.D. \\
			& 2nd supervisor \\
		\end{tabular}
	}
	\maketitle
	
	\section*{Abstract}
	Write an abstract
	
	\newpage
	
	%TODO Delete this comment for the final submission:
	\textcolor{red}{\textbf{!!! Prüfen Sie, dass der folgende Text aktuell ist (entsprechend der formalen Regeln des Studienbüros) !!! \\
	!!! Check that this text is up to date (according to formal rules of the examination office) !!!}}
	
	\section*{Erklärung zur Abschlussarbeit gemäß § 22 Abs. 7 APB TU Darmstadt}
	
	Hiermit versichere ich, \textcolor{red}{\textbf{Nikolaos Stivaktakis}}, die vorliegende Master-Thesis / Bachelor-Thesis gemäß § 22 Abs. 7 APB der TU Darmstadt ohne Hilfe Dritter und nur mit den angegebenen Quellen und Hilfsmitteln angefertigt zu haben. Alle Stellen, die Quellen entnommen wurden, sind als solche kenntlich gemacht worden. Diese Arbeit hat in gleicher oder ähnlicher Form noch keiner Prüfungsbehörde vorgelegen. 
	
	Mir ist bekannt, dass im Falle eines Plagiats (§38 Abs.2 APB) ein Täuschungsversuch vorliegt, der dazu führt, dass die Arbeit mit 5,0 bewertet und damit ein Prüfungsversuch verbraucht wird. Abschlussarbeiten dürfen nur einmal wiederholt werden.
	
	Bei einer Thesis des Fachbereichs Architektur entspricht die eingereichte elektronische Fassung dem vorgestellten Modell und den vorgelegten Plänen.
	
	\vspace{10pt}
	\hrule
	
	\section*{English translation for information purposes only:\\Thesis Statement pursuant to § 22 paragraph 7 of APB TU Darmstadt}
	
	I herewith formally declare that I, \textcolor{red}{\textbf{first name last name}}, have written the submitted thesis independently pursuant to § 22 paragraph 7 of APB TU Darmstadt. I did not use any outside support except for the quoted literature and other sources mentioned in the paper. I clearly marked and separately listed all of the literature and all of the other sources which I employed when producing this academic work, either literally or in content. This thesis has not been handed in or published before in the same or similar form.
	
	I am aware, that in case of an attempt at deception based on plagiarism (§38 Abs. 2 APB), the thesis would be graded with 5,0 and counted as one failed examination attempt. The thesis may only be repeated once.
	
	For a thesis of the Department of Architecture, the submitted electronic version corresponds to the presented model and the submitted architectural plans.
	
	\vspace{10pt}
	\hrule
	\vspace{70pt}
	
	\noindent\begin{tabular}{l@{\hskip 1in}l}
		\makebox[1.8in]{\hrulefill} & \makebox[3.5in]{\hrulefill}\\
		Datum / Date & Unterschrift / Signature
	\end{tabular}
	
	\tableofcontents
	
	\mainmatter
	
	\chapter{Introduction}

\chapter{Preliminaries}
\section{Cryptography}
\section{Game Theory}
\section{Blockchains}
\section{Smart Contracts}
\section{Escrows}

\chapter{Related work}
\section{first escrowless protocol}
Zimbeck mit Bithalo
auch die analyse erwähnen
\section{digital vs Physical}
Hier unbedingt Asganokar sein paper erwähnen (sale of digital goods)

\chapter{Formalization}
\section{Entities}

\subsection{Seller}An entity that owns a specific physical good $G$. The seller has bigger utility in owning $P$ amount of coins in contrast to owning $G$. He therefore desires to exchange $G$ against coins of amount $P$.

%Insert Utility function distribution
%u_s(p coins & good) > u_s(p coins) > u_s(good) > u_s(Nothing)

\subsection{Buyer}An entity that owns coins of at least amount of $P$. The Buyer has greater utility in owning $G$ in contrast to  owning $P$ amount of coins. He therefore desires to exchange coins of amount $P$ against $G$.

%Insert Utility function distribution
%u_b(p coins & good) > u_b(good) > u_b(p coins) > u_b(Nothing)

\subsection{Escrow/Smart Contract} An entity that should ensure that the trade is successful. It acts as a fully trusted oracle.

%TODO: Helper box protects the buyer from coercion by the seller
\subsection{Helper Box} Fully trusted by Buyer.  //need to expand on that

\section{The Game}

\subsection{Start Point}
A Seller and a buyer want to trade successfully: Both have funds in the given cryptocurrency. The seller is in possession of the physical good.

\subsection{End Point}
The buyer is in possession of the physical good and his funds have decreased by amount p. The funds of the seller have increased by p.

\subsection{Assumptions}
Both entities have access to the Escrow and know the addresses of each other. The seller also knows the physical address to which the buyer wants the good to be shipped. Other than this information they know nothing about each other.\newline Both parties can send funds in the given cryptocurrency to any address. The sender has the power to send/ship the physical good to any physical address he wants. The Escrow is a smart contract and both the seller and the buyer know the code of the Smart Contract.\newline
We assume both entities to behave rational, meaning that they both want to maximise their expected payoff.\newline
Transaction Fees and Fees for the escrows are not analysed.



\subsection{Success of the trade}
The trade is considered successful, if the situation transitions from the start point to the end point.

\subsection{Goal}
The goal is to design the Escrow in such a way, that it is the best strategy for both the buyer and the seller to perform a successful trade. That means that a successful trade should be the Nash Equilibrium.

%\subsection{Implementation}
%The underlying blockchain is immutable as long as there is an honest majority of %miners. We assume this honest majority.

\chapter{Proposed Possible Solutions}
\section{The naive approach}
The simplest protocol is the following: The seller ships the physical good to the provided address $A$ and the buyer sends coins of amount $P$ to the seller.  
%Here maybe a visualization
\subsection{Payout}
To analyze the payout for each party and each strategy I will use a table.\newline

\minisec{Example table}
\begin{center}
\begin{tabular}{ c||c|c| }
& H$_B$ & -H$_B$  \\
\hline
\hline
H$_S$ & $P_S | P_B$ & $P_S | P_B$ \\
\hline
-H$_S$ & $P_S | P_B$ & $P_S | P_B$ \\ 
\hline
\end{tabular}
\end{center}

\minisec{Description} On the top are the possible strategies for the buyer: $H_B$ is the strategy where the buyer behaves honestly and -H$_B$ is the strategy where the Buyer behaves dishonestly.\newline
On the left side are the strategies for the seller. Similar to the buyer, H$_S$ denotes the strategy where the seller behaves honestly and -H$_S$ denotes the strategy where the seller behaves dishonestly.\newline

For each combination of strategies the two parties get a specific payout. $P_S$ denotes the payoff of the seller and $P_B$ denotes the payoff of the buyer.
\newline
This protocol has an obvious downside: There is no incentive for the seller to ship the good as well as for the buyer to send the coins. \newline

It is important to note that we assume that sending the good will decrease the sellers payoff regardless if the buyer sends the coins or not.


\minisec{Payout for the naive approach}
\begin{center}
\begin{tabular}{ c||c|c| }
& H$_B$ & -H$_B$  \\
\hline
\hline
H$_S$ & $P | G$ & $0 | P+G$ \\
\hline
-H$_S$ & $P + G | 0$ & $G | P$ \\ 
\hline
\end{tabular}
\end{center}

\subsection{One Side collateral}
To incentivize an entity to behave honestly we introduce collateral. The collateral will be lost, if the entity does not behave honestly.  
\minisec{Seller Collateral}
If B sends the coins, S has to make the following decision: Does he send the good or not? We have to take into account, that B decides what happens with the collateral. If S does not behave honestly (does not send the good), B can punish S by reporting 0 to the Smart contract. This means, that the collateral would not be given back to S. What was designed to be an incentive for S to behave honestly, ended up as leverage for B over S. B can report 0 (and burn the collateral) even if S behaves honestly. Therefore B can use this leverage to coerce S into giving him extra resources: “Give me C/2 coins or I will report 0 to the Smart Contract and burn your collateral.“
Therefore it is best for S to not send the physical good.
Since not Sending is the best strategy, It is best for B to not even send P. The Nash Equilibrium would be in nobody sending anything and the trade not happening.
\minisec{Payout}
\begin{center}
\begin{tabular}{ c||c|c| }
& H$_B$ & -H$_B$  \\
\hline
\hline
H$_S$ & $P | G$ & $0 P- \frac{C}{2} | G + \frac{C}{2}$ \\
\hline
-H$_S$ & $P + G - c | 0$ & $G + P - \frac{C}{2} | \frac{C}{2}$ \\ 
\hline
\end{tabular}
\end{center}
\begin{equation}
    U_s(P) > U_s(nothing) 
\end{equation}
\newline
\chapter{Open Research/ Questions}

\chapter{Conclusion}
	
	
	%\bibliographystyle{}
	%\bibliography{}
	
	\appendix
\end{document}